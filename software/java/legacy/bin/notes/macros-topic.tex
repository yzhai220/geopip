\usepackage{amsmath}
\usepackage{amssymb}
\usepackage{epsfig}

\def\half{\frac{1}{2}}

\def\randomgraph{{\cal G}_{n,\frac{1}{2}}}
\def\scrp{{\mathscr P}}
\def\scrc{{\mathscr C}}
\def\scre{{\mathscr E}}

\newcounter{xxx}
\setcounter{xxx}{0}
\newcommand\XXX[1]{{\bf \em \addtocounter{xxx}{1} (\thexxx) [[#1]]}}

% \DeclareMathOperator{\deg}{deg} 



\def\Q{{\mathbb Q}}        % rationals
\def\F{{\mathbb F}}        % rationals
\def\Z{{\mathbb Z}}        % integers
\def\N{{\mathbb N}}        % naturals
\def\R{{\mathbb R}}        % reals
\def\C{{\mathbb C}} 	   % complex
\def\Rn{{\R^{n}}}          % product of n copies of reals
\def\Cf{{\mathbf C}}	   % continuous functions on (.)

\def\P{{\mathbb P}}        % probability
\def\E{{\mathbb E}}        % expectation 
\def\1{{\mathbf 1}}        % indicator
\def\var{{\mathop{\mathbf Var}}}    % variance

\def\done{\hfill\rule{2mm}{2mm}}

\def\borel{{\cal B}}    % borel sigmal field
\def\G{{\cal G}}        % some sigma field
\def\F{{\cal F}}        % some sigma field
\def\H{{\cal H}}        % some sigma field

\def\L{{\mathbf L}}     % L, as in L^2

\def\ascv{\stackrel{\scriptscriptstyle a.s.}{\longrightarrow}}     % almost sure convergnece
\def\pcv{\stackrel{\scriptscriptstyle \P}{\longrightarrow}}        % convergence in P
\def\ltcv{\stackrel{\scriptscriptstyle\L^2}{\longrightarrow}}      % L2 convergnece
\def\lpcv{\stackrel{\scriptscriptstyle\L^p}{\longrightarrow}}      % Lp convergnece
\def\dcv{\stackrel{\scriptscriptstyle d}{\longrightarrow}}         % convergence in d
\def\deq{\stackrel{\scriptscriptstyle d}{=}}			   % equal in d
\def\toinf{\to \infty}

\def\ci{\perp\!\!\!\perp}  % conditional independence

% the header/lecture structure...borrowed from alistair sinclair's cs270 format
% \newcommand{\topic}[3]{
%    \pagestyle{myheadings}
%    \thispagestyle{plain}
%    \newpage
%    \setcounter{page}{1}
%    \noindent
%    \begin{center}
%    \framebox{
%       \vbox{\vspace{2mm}
%     \hbox to 6.28in { {\bf STAT 205~Probability Theory
%                         \hfill Fall 2006} }
%        \vspace{4mm}
%        \hbox to 6.28in { {\Large \hfill Topic: #1 \hfill} }
%        \vspace{2mm}
%        \hbox to 6.28in { {\it Lecturer:} Jim Pitman, {\it Scribe:} #2, {\it Editor:} #3 \hfill }
%       \vspace{2mm}}
%    }
%    \end{center}
%    \markboth{Topic: #1}{Topic: #1}
%    \vspace*{4mm}
% }
       %\hbox to 5.90in { \hfill {\it Scribe: #3} }


% \newtheorem{theorem}{Theorem}
% \newtheorem{lemma}[theorem]{Lemma}
% \newtheorem{proposition}[theorem]{Proposition}
% \newtheorem{claim}[theorem]{Claim}
% \newtheorem{corollary}[theorem]{Corollary}
% \newtheorem{definition}[theorem]{Definition}
% \newenvironment{proof}{{\bf Proof:}}{\hfill\rule{2mm}{2mm}}
% \newenvironment{proofsketch}{{\bf Proof Sketch:}}{\hfill\rule{2mm}{2mm}}
% \newcommand{\fig}[3]{
%       \vspace{#2}
%       \begin{center}
%       Figure #1:~#3
%       \end{center}
% }
% \newtheorem{exercise}[theorem]{Exercise}
% \newtheorem{example}[theorem]{Example}

% %some things we'll never understand....alas
% \setlength{\oddsidemargin}{0.25 in}
% \setlength{\evensidemargin}{-0.25 in}
% \setlength{\topmargin}{-0.6 in}
% \setlength{\textwidth}{6.0 in}
% \setlength{\textheight}{8.5 in}
% \setlength{\headsep}{0.75 in}
% \setlength{\parindent}{0 in}
% \setlength{\parskip}{0.1 in}
% \advance\topmargin by 0.5in

